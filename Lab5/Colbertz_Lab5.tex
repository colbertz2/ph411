\documentclass[11pt]{article}
    \usepackage{comment} % enables the use of multi-line comments (\ifx \fi) 
    \usepackage{lipsum} %This package just generates Lorem Ipsum filler text. 
    \usepackage{fullpage} % changes the margin
    % Used for importing figures
    \usepackage{graphicx}
    \usepackage{wrapfig}
    \usepackage{float}
    \usepackage{subcaption}
    
\begin{document}

%Header-Make sure you update this information!!!!
\noindent
\large\textbf{Lab 5} \hfill \textbf{Zach Colbert} \\
\normalsize PH 411 \hfill 21 November 2017\\

\section{Transistor Basics}
\subsection{Introduction}

Transistors are active elements--they control current by means of a \"base\" signal, separate from the main signal passing through. Junctions between semiconductors make this possible, where electron-hole pairs come together to form a charge imbalance called the \"depletion zone,\" effectively a barrier which prevents electrons from flowing across the junction.\\

By applying some current to the base of the transistor, connected to the middle semiconductor, we suppress the depletion zones enough to open a kind of channel for current to flow through.\\

In addition to acting as electronic switches, starting and stopping current with a small base signal, transistors come with some current gain--by which the input signal is amplified with a still small base singal.\\

In part 1 of this lab, we examine this current gain property of an npn transistor, by two different methods.\\

\subsection{Experimental}

\begin{figure}[H]
    \centering
    \includegraphics[scale=0.4]{Diagrams/c-1c.png}
    \caption{Simple circuit with an npn transitor.}
    \label{circuit:1c}
\end{figure} 

For this part of the lab, we used a 2N3904 npn transistor. Before looking at the current gain, we used a digital multimeter on the diode setting to show that a transistor behaves like two diodes. We were able to measure voltage drops across the \"diodes\" on either side of the transistor (base-emitter and collector-base drops) as well as the drop across the entire transistor (collector-emitter drop).\\

Next, as a point of comparison for our current gain, we used the DMM on the transistor gain setting to measure our transistor on its own.\\

Finally, for a little more comprehensive look at the behavior of the transistor over a range of currents, we built the transistor into a circuit and measured the base and collector currents individually.\\

In the above circuit, we used a $1\ M \Omega$ potentiometer to manually adjust the base current, and compared that to the collector current over many samples (keeping the respective input voltages at the collector and base constant throughout). Our theoretical model for the current gain says:

\begin{equation}
I_c = \beta I_b
\end{equation} 

So, we expect a plot of $I_c$ versus $I_b$ to be a line with slope $\beta$, where $\beta$ is the current gain of the transistor.\\

\subsection{Results}

In measuring the \"diode\" drop voltages across parts of the transistor, we found the following: \\

\begin{center}
    \begin{tabular}[H]{ | l | c | }
        \hline
        Base-Emitter & $0.665\ V$ \\ \hline
        Collector-Base & $0.646\ V$ \\ \hline
        Emitter-Collector & $0\ V$ \\ \hline
    \end{tabular}
\end{center}

The first two measurements showed us that current flows one way through each side of the transistor, just like current flows one way through a diode. The third showed the current-controlling function of the transistor--without a base current, no current flows between the collector and emitter.\\

\begin{figure}[H]
    \centering
    \includegraphics[scale=0.4]{Plots/fig1.png}
    \caption{Current gain of an npn transistor, measured via various methods.}
    \label{fig:1}
\end{figure}

With the DMM on transistor mode, we measured a current gain $\beta = 112$, shown in green on Figure \ref{fig:1}.\\

Our manually measured gain was incredibly consistent, following a distinct line until about $60\ \mu A$ base current. At that point it curved away and became shallow--By some simple analysis of the circuit, this appears to be a maximum for the collector current. \\

That final point lies at about $15\ mA$ collector current. Through a $1\ k \Omega$ resistor, that current results in a $15\ V$ drop between the constant collector voltage $V_{cc}$ and the collector. Not only is that the maximum voltage that can be passed to the collector, it causes the collector voltage to be 0, equal to the emitter voltage. This breaks one of the fundamental rules for normal operation of a transistor.\\

I added a couple of different linear fits to the plot--the line in blue represents a linear fit across all of the data. The line in red represents a linear fit across a select range of data, excluding the points above $60\ \mu A$ base current.\\

The inclusive linear fit represents a current gain of 144, and the exclusive linear fit represents a current gain of 167. I haven't found a good explanation for the difference between these values and the one measured by the DMM initially, but they seem to be reasonably close to each other (at least, within an order of magnitude).\\

\subsection{Conclusion}

There are clear differences between the values measured for current gain in this part of the lab, but it seems reasonable to assume that measurements exclusively made by the DMM are not the most accurate.\\

In this case, if I wanted to characterize this transistor for a larger project, I would tend to choose the current gain found by making a linear fit in the region where the data appears to be linear. Then, at least it is clear what the current gain is for that region, and my larger circuit can be constrained to fit the parameters for normal operation of the transistor.\\
    
%%% PART 2 %%%

\section{Emitter-Follower}
\subsection{Introduction}

\subsection{Experimental}

\subsection{Results}

\subsection{Conclusion}

%%% PART 3 %%%

\section{Transistor Switch}
\subsection{Introduction}

\subsection{Experimental}

\subsection{Results}

\subsection{Conclusion}


\end{document}
